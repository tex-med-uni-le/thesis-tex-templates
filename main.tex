\documentclass[a4paper]{book}


\usepackage{geometry} % Titelseite braucht eigene Margins.

\usepackage[ngerman]{babel} % Trennungsregeln, Datumsangabe etc. in Deutsch

\usepackage[T1]{fontenc}    % europäischer Zeichensatz (Sonderzeichen, etc.)
\usepackage[bitstream-charter]{mathdesign} % Moderne, serifenbetonte und sachliche Schriftart
\usepackage[utf8]{inputenc} % utf8 Eingabezeichensatz

\usepackage{xstring}        % Testen/Vergleichen von (Sub)Strings

\usepackage{totcount}       % Zähle Seitenzahlen, Abbildungen etc.

\usepackage{siunitx}        % SI Einheiten
\usepackage{natbib}         % Naturwissenschaftliche Zitierungen
\usepackage[printonlyused]{acronym} % Abkürzung + Abkürzungsverzeichnis

\usepackage{blindtext}      % Seitenfüller zum Testen

%% Eigene Makros
%% Konfigurationsdatei

% Name
\newcommand{\tAuthor}{Max Mustermann}
% Titel der Dissertation
\newcommand{\tTitle}{Musterpromotion}
% zu erlangender Grad: Dr. med./Dr. med. dent./Dr. rer. med.
\newcommand{\tDegree}{Dr. med.}
% Geburtstag
\newcommand{\tBirthday}{02.12.1409}
% Geburtsstadt
\newcommand{\tNativeTown}{Leipzig}
% Hochschule/Institut
\newcommand{\tDepartment}{Universtität Leipzig / Klinik für \TeX-Krankheiten}
% Betreuer
\newcommand{\tSupervisor}{Carl Ludwig \\ Paul Flechsig}
% Datum der Verteidigung
\newcommand{\tDefenceDate}{01.01.1970}
% Monat der Einreichung
\newcommand{\tSubmissionMonth}{January}
% Jahr der Einreichung
\newcommand{\tSubmissionYear}{1970}

%% \InOderAm
%% Sucht nach "Klinik" in einem String und gibt im Erfolgsfall "in der"
%% ansonsten "am" zurück.
%% params: string
%% returns: "in der"/"am"
%%
\newcommand{\InOderAm}[1]{
  \expandarg
  \IfSubStr{#1}{Klinik}{in~der}{am}
}

%% \HideIfZero
%% Zeigt den Text (#2) nur, wenn der counter (#1) größer 0 ist.
%% params: 1: counter; 2: text
%% returns: nichts/text
\newcommand{\HideIfZero}[2] {
  \ifnum\totvalue{#1}>0%
  #2%
  \else%
  \null%
  \fi%
}

%% Definiere Zähler für Zitierungen:
%% http://texblog.org/2012/04/16/counting-the-total-number-of/
\newtotcounter{citenum}
\def\oldcite{}
\let\oldcite=\bibcite
\def\bibcite{\stepcounter{citenum}\oldcite}


%% Dokument
\begin{document}

%% Titelblatt
\begin{titlepage}
\newgeometry{margin=1.5in}
\begin{center}

\null
\vspace{\baselineskip}
{\huge {\bfseries \tTitle \par}}
\vspace{5\baselineskip}
DISSERTATION \\
\vspace{\baselineskip}
zur Erlangung des akademischen Grades \\
\vspace{\baselineskip}
\tDegree \\
\vspace{\baselineskip}
an der\\ \vspace{.2\baselineskip}{\large \bf Medizinischen Fakultät der Universität Leipzig} \\
\vspace{6\baselineskip}

\vfill

    \noindent
    eingereicht von \\
    \vspace{.75\baselineskip}
    {\bf \tAuthor} \\
    \vspace{.25\baselineskip}
    geboren am \tBirthday{} in \tNativeTown \\

\vspace{3\baselineskip}

    \noindent
    angefertigt \InOderAm{\tDepartment} \\
    \tDepartment \\

\vspace{1.5\baselineskip}

    \noindent
    Betreut von \\
    \vspace{.5\baselineskip}
    \tSupervisor \\

\vspace{1.5\baselineskip}

    \noindent
    Beschluss über die Verleihung des Doktorgrades vom \tDefenceDate

\end{center}
\restoregeometry
\end{titlepage}



%% Zeilennummer für die Inhaltsverzeichnis und Co.
\setcounter{page}{1}
\pagenumbering{Roman}

%% TOC
\tableofcontents
\newpage

\pagenumbering{arabic}
%% Bibliographische Beschreibung
\clearpage

\section{Bibliographische Beschreibung}

%% Lade offizielle bibliographische Beschreibung
\null
\vspace{2\baselineskip}
\noindent
\tAuthor \\
\vspace{2\baselineskip}
\noindent \\
\tTitle \\
\vspace{2\baselineskip}
\noindent \\
Universität Leipzig, Dissertation \\
\vspace{2\baselineskip}

%% Registriere verschieden Zähler
\regtotcounter{page}
\regtotcounter{totalfigures}
\regtotcounter{table}
\regtotcounter{totalappendix}

\noindent
\HideIfZero{page}{\total{page}~Seiten, }%
\HideIfZero{citenum}{\total{citenum}~Literaturangaben, }%
\HideIfZero{totalfigures}{\total{totalfigures}~Abbildungen, }%
\HideIfZero{table}{\total{table}~Tabellen, }%
\HideIfZero{totalappendix}{\total{totalappendix}~Anhänge}%
\vspace{5\baselineskip}


\noindent
%% Beginn der eigenen Zusammenfassung/Referat
\blindtext % Entferne diese Zeile

\clearpage

\section{Abkürzungsverzeichnis}

\begin{acronym}[MMM] %% Längstes Acronym hier in die eckigen Klammern schreiben
%% Ab hier anpassen:
  \acro{CLI}{Carl-Ludwig-Institut}
  \acro{MaP}{Mensa am Park}
  \acro{UKL}{Uniklinikum Leipzig}
\end{acronym}

\newpage

%% Hauptteil
% Hier kann alles ersetzt werden:
\chapter{Kapitel}

\section{Abschnitt}

\subsection{Unterabschnitt}
siunitx test: \SI{123}{\pico\gram\per\mole}

\blindtext

\subsubsection{Zitiertest:}
Das R-Project zitiert mit citep \citep{RPROJECT} \dots \\
Das R-Project nochmal anders mit citet \citet{RPROJECT}. \\
Wir zitieren auch noch \citep{KM80}. \\


%% Schlussteil

%% Zusammenfassung
\clearpage

\chapter{Zusammenfassung}

%% Lade offizielle Zusammenfassung
\null
\vspace{2\baselineskip}
\noindent
Dissertation zur Erlangung des akademischen Grades: \\
\tDegree \\
\vspace{\baselineskip}
\noindent \\
\tTitle \\
\vspace{\baselineskip}

\noindent
eingereicht von: \\
\tAuthor \\

\noindent
angefertigt \InOderAm{\tDepartment}: \\
\tDepartment \\

\noindent
betreut von: \\
\tSupervisor \\

\noindent
\tSubmissionMonth{}~\tSubmissionYear \\
\vspace{5\baselineskip}


\noindent
%% Beginn der eigenen Zusammenfassung
\blindtext % Entferne diese Zeile

%% Literaturverzeichnis
\cleardoublepage

% http://tex.stackexchange.com/questions/26225/added-chapter-with-addcontentsline-does-not-adapt-style-of-toc
\renewcommand{\bibsection}{\chapter{\bibname}}

\bibliographystyle{apalike}
\bibliography{bibliography/bibliography}

%% Appendix
\appendix

\stepcounter{totalappendix}
\clearpage
\thispagestyle{empty}

\section*{Erklärung über die eigenständige Abfassung der Arbeit}

Hiermit erkläre ich, dass ich die vorliegende Arbeit selbständig und ohne unzulässige Hilfe
oder Benutzung anderer als der angegebenen Hilfsmittel angefertigt habe. Ich versichere,
dass Dritte von mir weder unmittelbar noch mittelbar geldwerte Leistungen für Arbeiten
erhalten haben, die im Zusammenhang mit dem Inhalt der vorgelegten Dissertation stehen,
und dass die vorgelegte Arbeit weder im Inland noch im Ausland in gleicher oder ähnlicher
Form einer anderen Prüfungsbehörde zum Zweck einer Promotion oder eines anderen
Prüfungsverfahrens vorgelegt wurde. Alles aus anderen Quellen und von anderen
Personen übernommene Material, das in der Arbeit verwendet wurde oder auf das direkt
Bezug genommen wird, wurde als solches kenntlich gemacht. Insbesondere wurden alle
Personen genannt, die direkt an der Entstehung der vorliegenden Arbeit beteiligt waren.

\vspace{4\baselineskip}

%% Datum und Unterschrift
\begin{center}
\begin{minipage}[t]{0.48\textwidth}
Leipzig, den \today
\end{minipage} % Kein "newline" hier
\begin{minipage}[t]{0.48\textwidth}
\vspace{0.1\baselineskip}
\rule{12em}{0.5pt} \\
\tAuthor
\end{minipage}
\end{center}





\end{document}
